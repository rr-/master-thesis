\section{Metody obrony}

Z~uwagi na naturę ataków, wyróżnia się dwie podstawowe uniwersalne linie obrony
przed złamaniem haszy.

\subsection{Ataki \en{online}}

W~przypadku ataków \en{online} tak naprawdę mamy do czynienia tylko z~jedną
metodą obrony~-- jest nią identyfikacja atakującego i~następujące po niej
zablokowanie dostępu do danej usługi.

Identyfikacja może następować na różne sposoby. Administrator mając do
dyspozycji logi z~nieudanych prób uwierzytelnienia może szukać pewnych
korelacji poprzez np.:

\begin{itemize}

\item porównywanie adresów IP z~których pochodziły kolejne nieudane próby,

\item sprawdzanie obecności i~poprawności pewnych ustalonych informacji
przechowywanych w~metadanych połączenia (np. ciasteczkach przeglądarki).

\end{itemize}

Kiedy atakujący zostanie zidentyfikowany jako osoba, która już wcześniej
wielokrotnie podejmowała nieudane próby uwierzytelnienia, system może
odpowiednio zareagować. Jedną z~najpopularniejszych metod reakcji jest
ograniczenie liczby możliwości wypróbowania kolejnych kombinacji w~ustalonym
odcinku czasu. Przykładem może być formularz logowania, w~którym w~ciągu 10
minut nie można dokonać więcej niż 5~prób zalogowania. Jest to proste i~często
spotykane zabezpieczenie, jednak tak naprawdę nie gwarantuje ono praktycznie
żadnego bezpieczeństwa.

Ataki \en{online} mogą być przeprowadzane z~jednego komputera. Atak tego typu
jest zazwyczaj mało skuteczny z~powodu ograniczeń przepustowości sieci,
a~zidentyfikowanie atakującego przeprowadzającego kolejne próby
uwierzytelnienia z~jednego adresu IP jest trywialne. Odmowa dostępu działa więc
tutaj bardzo dobrze. Należy jednak zwrócić uwagę, że atak brutalny
przeprowadzany z~jednego komputera w~zasadzie nie różni się wiele pod względem
technologicznym od ręcznego sprawdzania popularnych loginów i~haseł przez
złośliwego użytkownika.

Ataki \en{online} nabierają tak naprawdę znaczenia dopiero w~sytuacji, kiedy
atakujący:

\begin{itemize}

\item korzysta z~wielu komputerów i/lub
\item korzysta z~serwerów proxy i/lub
\item korzysta z~sieci TOR.

\end{itemize}

Im lepiej wyposażony jest atakujący, tym identyfikacja źródła jest trudniejsza,
a~więc coraz trudniejsze staje się efektywne odebranie dostępu atakującemu
do danej usługi. Można bezpiecznie założyć, że poważny atakujący będzie
dysponował wszystkimi tymi zasobami.

Tak naprawdę problem zapobiegania atakom o~większej skali sprowadza się do
zapobiegania atakowi \en{offline}, w~którym system nie ma możliwości efektywnej
odmowy dostępu, a~atakujący nie jest ograniczony przepustowością sieci.
Dodatkowo ataki tego typu cechują własności ataku DDoS, przed którego obroną
należy się dodatkowo zabezpieczyć osobnymi sposobami.

\subsubsection{Obrona przed atakami DDoS}
W~kontekście kryptograficznych funkcji skrótu istnieje kilka sytuacji,
w~których można mówić o~ataku DDoS.

\begin{itemize}

\item Kryptograficzna funkcja haszująca działa wystarczająco powoli, aby nawet
niewielka liczba zapytań wyczerpała zasoby serwera.

\item ,,Klasyczny'' atak DDoS~-- zasoby serwera są wyczerpane przez dużą liczbę
zapytań rozsyłanych z~wielu lokacji.

\item Nieprzemyślane wykorzystanie funkcji haszujących: przykładem może być
tablica mieszająca; w~przeciętnym przypadku pozwala ona wyszukiwać elementy
w~czasie $O(1)$, ale w~najgorszym przypadku może działać w~czasie $O(n)$.
Atakujący może znać sposób działania danego mechanizmu na serwerze
i~przygotowywać specjalne zapytania, które będą powodować powolne działanie
użytych struktur, co ułatwia wyczerpanie zasobów serwera~\cite{ddos_hashes}.
Atak ten rzadko jest związany z~\emph{kryptograficznymi} funkcjami haszującymi,
należy jednak zwrócić uwagę, że także one są czasem wykorzystywane w~tego typu
zastosowaniach (np. system git używa \texttt{SHA-1} do identyfikacji
konkretnych zmian kodu).

\end{itemize}

Metod zabezpieczeń przed tego typu atakami w~zasadzie nie ma zbyt wiele.
W~przypadku słabości wynikających z~(pośredniego lub bezpośredniego) powolnego
działania funkcji haszujących można próbować zastąpić je szybszymi funkcjami
lub polepszać algorytmy z~nich korzystające. Nie zawsze jest to jednak możliwe;
wówczas można starać się zidentyfikować zapytania potencjalnie powodujące
powolne działanie algorytmów i~przekształcać je na bezpieczne odpowiedniki.
Jednak także taka linia obrony nie zawsze jest możliwa do przeprowadzenia.

W~sytuacji, kiedy dany zasób na serwerze zawsze będzie reagował wolno i~nie
można nic na to poradzić, sytuacja sprowadza się w~zasadzie do zabezpieczenia
serwera przed klasycznymi atakami DDoS. Jedyna metoda skutecznej obrony to
stosowanie firewalli odcinających dostęp atakującym oraz load-balancing, czyli
rozpraszanie przychodzącego ruchu na wiele serwerów. Przykładami zewnętrznych
firm, które świadczą dedykowane usługi w~zakresie obrony przed atakami DDoS,
mogą być \en{Incapsula} i~\en{Distil}~-- zasada ich działania sprowadza się do
roli potężnego proxy pomiędzy użytkownikiem a~serwerem udostępniającym zasoby.

\subsection{Ataki \en{offline}}

Atak \en{offline} ma miejsce wówczas, kiedy atakujący może wypróbowywać do woli
kolejne kombinacje z~efektywną prędkością bez obawy przed wykryciem. Istnieje
kilka metod, które pozwalają się zabezpieczyć przed tego typu atakami lub
przynajmniej w~znacznym stopniu je utrudnić.

\subsubsection{Wymuszanie losowości tekstu wejściowego (serwisy internetowe)}
Mnóstwo serwisów internetowych wymusza na użytkownikach stosowanie haseł
spełniających jedną i/lub więcej poniższych cech:

\begin{enumerate}
\item hasło jest długie na co najmniej $n$ znaków,
\item hasło ma przynajmniej jedną wielką literę,
\item hasło ma przynajmniej jedną cyfrę,
\item hasło ma przynajmniej jedną literę ,,specjalną''.
\end{enumerate}

Podejście to jest niewygodne dla użytkowników. Tak naprawdę bezpieczeństwo
serwisu nie powinno być zależne od tego, czy jakiś użytkownik będzie miał hasło
postaci \texttt{qwerty}; co więcej, również sam użytkownik stosujący takie
hasło powinien być względnie bezpieczny w~sytuacji, kiedy baza danych
z~hasłami zostanie wykradziona.

\subsubsection{Konkatenacja haszy}

Jednym z~pomysłów jest próba zwiększenia siły funkcji skrótu bez jej
faktycznego zmieniania przez zastosowanie konkatenacji. Zamiast przechowywać
w~bazie hasz postaci:
    $$H(m)$$
przechowywane są hasze postaci:
    $$H_1(m)||H_2(m)||\ldots||H_n(m)$$
gdzie $H_i$ to kolejne, różne od siebie, funkcje haszujące. Dawne protokoły
SSL/TLS korzystały z~$\mathtt{MD5}(m)||\mathtt{SHA1}(m)$. Konstrukcja ma na
celu sprawienie, że nawet jeżeli atakujący znajdzie $m' : H_i(m') = H_i(m)$, to
wykorzystanie innych $H_{j \neq i}$ zagwarantuje że cały hasz i~tak będzie
różny, gdyż prawdopodobieństwo $\forall_{j \neq i} \; H_j(m') = H_j(m)$ będzie
minimalne.

Okazuje się jednak, że konstrukcja tego typu dla haszy opartych o~konstrukcję
Merkle-Damg\r{a}rda jest tak samo odporna na \en{collision attacks} jak
najsilniejsza funkcja $H_i$ wykorzystana w~ciągu~\cite{md5_concatenation}.
Dodatkowo atakujący potrafiąc generować $m_1, m_2 : m_1 \neq m_2 \wedge
\mathtt{MD5}(m_1) = \mathtt{MD5}(m_2)$, potrafi wygenerować łatwo $m_3, m_4,
\ldots, m_n : \forall_{1 \leq i \leq n} \; H(m_i) =
H(m_1)$~\cite{md5_multi_collision_attack}. Zabezpieczający system powinien
zatem zrezygnować z~tego podejścia na rzecz lepszych metod.

\subsubsection{Potęgowanie haszy}

Inne rozwiązanie stanowi podejście, w~którym zamiast przechowywać w~bazie skrót
postaci:
    $$H(m)$$
przechowywane są hasze postaci:
    $$H(H(\ldots H(m) \ldots)) = H^{(n)}(m)$$
Spowoduje to, że czas potrzebny na obliczenie pojedynczego skrótu wydłuży się
nawet $n$-krotne~--~a tym samym spowoduje, że atakujący będzie musiał poświęcić
więcej czasu na każdą wypróbowywaną kombinację. Prostą kontrolę nad czasem
potrzebnym na obliczenie haszu stanowi liczba funkcji użytych do złożenia
($n$).

Podejście to, mimo że na pierwszy rzut oka atrakcyjne (przy $n = 20000$
atakujący potrzebuje nawet 20000x więcej czasu na znalezienie kolizji!), ma
duże wady z~następujących powodów:

\begin{itemize}

\item Większość kryptograficznych funkcji haszujących takich jak \texttt{MD5}
oraz \texttt{SHA-1} daje się obliczać błyskawicznie. Liczba $n$ musi być
naprawdę wysoka, aby modyfikacja ta miała sens.

\item Konstrukcja takich funkcji jest niezbadana. Przykładowo, nie wiadomo, czy
$\mathtt{MD5}^{(n)}(m)$ nie jest zbieżne do wartości przejawiających przewidywalne
cechy. Stosowanie niezbadanych konstrukcji w~zastosowaniach kryptograficznych
jest stanowczo odradzane; historia pokazuje, że stosowanie mechanizmów, które
nie zostały poddane dogłębnej kryptoanalizie może okazać się katastrofalne.
Przykładem może być np.~\cite{untested_cryptography}.

\item Znaczne wydłużanie czasu potrzebnego na obliczenie haszu powoduje, że
zasób dokonujący uwierzytelniania, jeżeli jest dostępny \en{online}, staje się
podatny na ataki DDoS~-- przy wolno działającym mechanizmie uwierzytelniania
potrzeba znacznie mniej zapytań, aby wyczerpać zasoby serwera.

\end{itemize}

\label{salt_1}%
\subsubsection{,,Sól'' (metoda I)}

Sól stanowi dobrą linię obrony przed atakami \en{offline}. Ma ona na celu
zagwarantowanie nieskuteczności ataków przy pomocy tęczowych tablic oraz ataków
na więcej niż 1~hasło naraz.

Wyobraźmy sobie scenariusz, w~którym w~bazie danych hasła użytkowników
przechowywane są w~postaci $H(p_u)$, gdzie $p_u$ to hasło użytkownika $u$.
Atakujący dysponując wiedzą o~zawartości takiej bazy danych może skonstruować
tablice tęczowe, które pozwolą mu przeprowadzić szybki atak słownikowy oraz
sprawdzać obecność popularnych skrótów, takich jak np. $H(\texttt{1234})$ czy
$H(\texttt{qwerty}$). Techniki te bardzo szybko doprowadzą atakującego do
odgadnięcia haseł potencjalnie wielu użytkowników.

Aby zabezpieczyć się przed tego typu atakiem, można wprowadzić ,,sól'', czyli
unikalny ciąg losowych znaków. Zamiast przechowywać w~bazie $H(p)$,
przechowywane są dwie wartości: $H(p_u || s_u)$ oraz niezaszyfrowane $s_u$,
gdzie $p_u$ to hasło użytkownika $u$, a~$s_u$ to ,,sól'' wygenerowana dla
użytkownika $u$. Ważne jest, aby zachodziła poniższa własność:
    $$\forall_{u_1, u_2 \in U} \; s_{u_1} \neq s_{u_2}$$
Uwierzytelnienie hasła $p'$ będzie przebiegało następująco:

\begin{itemize}
\item pobierz z~bazy $s_u$ dla zadanego użytkownika $u$,
\item oblicz $H(p'||s_u)$,
\item sprawdź, czy $H(p'||s_u) = h_u$, gdzie $h_u$ to przechowywany w~bazie
$H(p_u||s_u)$.
\end{itemize}

Atakujący wówczas chcąc złamać $h_{u_k}$ musiałby skonstruować specjalne
tablice tęczowe dla funkcji haszującej $H(x||s_{u_k})$. Tak otrzymane tablice
byłyby bezużyteczne w~przypadku sprawdzania każdego innego użytkownika,
ponieważ
    $$\forall_{j \neq k} \; s_{u_j} \neq s_{u_k} \rightarrow H(x||s_{u_j}) \neq
    H(x||s_{u_k})$$
Oczywiście jeżeli atakujący skupia się tylko na jednym użytkowniku, to
wprowadzenie soli nie spowalnia takiego ataku~-- nabiera ona znaczenia jedynie
kiedy atakujący stara się wykraść obojętnie jakie hasło.

Dodatkową własnością wynikającą z~powyższego rozwiązania jest wydłużenie czasu
potrzebnego na sprawdzenie obecności pojedynczego hasła w~bazie z~1 operacji
haszowania do $n$ operacji haszowania, gdzie $n$ stanowi liczbę użytkowników
w~bazie (np. chcąc sprawdzić, czy istnieje użytkownik z~hasłem \texttt{1234},
trzeba dla każdego użytkownika $u$ obliczyć $H(\texttt{1234}||s_u)$). Należy
jednak zwrócić uwagę że przyrost ten nie polepsza znacząco bezpieczeństwa;
przetestowanie takich haseł nadal będzie się odbywało bardzo szybko.

\label{salt_2}%
\subsubsection{,,Sól'' (metoda II)}
Drugim rodzajem zabezpieczenia korzystającego z~soli jest stosowanie utajnionej
soli. Zabezpieczenie to, w~przeciwieństwie do poprzedniej metody, zakłada, że
taka sól będzie wspólna dla wszystkich użytkowników, a~więc zamiast
przechowywać $H(p_u)$ przechowywany jest $H(p_u||s)$. Podczas uwierzytelniania
$p'$ sprawdzane jest, czy $H(p'||s) \stackrel{?}{=} H(p_u||s)$.

Podstawową korzyścią jest bezpieczeństwo w~sytuacji, gdy baza została
wykradziona. Ponieważ $s$ nie jest przechowywane w~bazie, atakujący nie może
przeprowadzić żadnego sensownego ataku~-- tak naprawdę nie zna nawet $H$, na
jakie mógłby przeprowadzić atak. Nawet dysponując wiedzą, że użyte zostało np.
$\texttt{MD5}$, nie posiada znaczącej przewagi, gdyż do przeprowadzenia ataku
brutalnego potrzebuje dodatkowo nieznanego $s$. Jeżeli $s$ jest wystarczająco
długie, to jest ono bezpieczne przed atakiem brutalnym.

Drugą korzyścią płynącą z~użycia tej linii obrony jest to, że jej zastosowania
są nieco bardziej uniwersalne. Przykładowo można weryfikować, czy dane
przesłane zewnętrznemu podmiotowi nie uległy zmianie, czyli np. czy ciastka
które kazaliśmy pamiętać przeglądarce nie zmieniły się między kolejnymi
zapytaniami (ciastko zawiera dane $x$ i~skrót $h = H(x||s)$, gdzie $s$ to
sekretna sól; poprawność $h$ może później zostać zweryfikowana po stronie
serwera).

Bezpieczeństwo takiego systemu w~całości opiera się na tajności $s$. Wiąże się
to niestety z~kilkoma wadami.
\begin{enumerate}

\item Należy zwrócić szczególną uwagę na sposób łączenia wejściowych danych
z~solą. Jeżeli łączenie będzie się odbywało poprzez $s||m$, to wówczas nasza
funkcja będzie podatna na \en{length extension
attack}~(\ref{sec:length_extension_attack}).

\item System wymaga zaufania przynajmniej jednej osobie, w~szczególności
każdemu, kto ma uprawnienia do czytania $s$ po stronie serwera (a więc np.
administratorowi).

\item W~sytuacji kiedy $s$ zostanie wykradzione osoba, która uzyskała dostęp do
$s$, ma najprawdopodobniej dostęp także do wszystkich innych zasobów (takich
jak np. baza danych zawierająca $h=H(m||s)$). Wówczas może ona bez problemu
przeprowadzić atak brutalny konstruując tablice tęczowe dla funkcji $H(x||s)$.
Dobrym zabezpieczeniem byłoby użycie hybrydowego rozwiązania, stosując
$H(p_u||s||s_u)$, gdzie $s$ to tajna sól, $p_u$ to hasło użytkownika $u$
a~$s_u$ to publiczna sól użytkownika $u$. Wówczas bezpieczeństwo systemu po
wykradzeniu $s$ byłoby równoważne bezpieczeństwu systemu opisanego w~podejściu
~\ref{salt_1}.

\item Nie da się zmienić $s$ na $s'$ bez obliczania $\forall_{m \in M} \;
H(m||s')$ na nowo. W~sytuacji kiedy nie dysponuje się oryginalnym $m$, nie jest
to możliwe. Jeżeli przechowujemy hasła użytkowników jedynie za pomocą ich
haszy, w~przypadku konieczności zmiany $s$ należy wymusić na użytkownikach
zmianę hasła, a~to jest z~kolei niemile widziane.

\end{enumerate}
Jeżeli jednak jesteśmy w~stanie zagwarantować bezpieczeństwo $s$ (np. poprzez
zastosowanie w~tym celu osobnego systemu kryptograficznego), podejście to
w~połączeniu z~zastosowaniem~\ref{salt_1} stanowi najlepsze z~powyższych.

\subsubsection{\en{Key Derivation Functions}}
Powyższe sposoby zabezpieczenia się przed atakami \en{online} oraz \en{offline}
są uogólnione przez mechanizm \en{Key Derivation Function}. Funkcje z~tej
rodziny służą do tworzenia pochodnego klucza lub kluczy na podstawie tajnego
argumentu. Przykładem implementacji \abbr{KDF} są właśnie kryptograficzne
funkcje skrótu wykorzystujące sól. Ogólna postać takiej funkcji wyraża się
wzorem
    $$K' = \mathrm{KDF}(K, S, n)$$
gdzie $K'$ to otrzymany klucz, $K$ to tajny klucz źródłowy, $S$ to sól, a~$n$
to liczba iteracji mająca wydłużyć działanie funkcji w~celu udaremnienia ataków
siłowych.

\subsubsection{\en{Key Stretching}}
Wspomniana wcześniej metoda potęgowania haszy również znalazła swoje
uogólnienie w~postaci tzw. \en{Key Stretching}. Technika ta ilustruje, w~jaki
sposób można konstruować \en{Key Derivation Functions}; poniżej zamieszczono
kilka przykładów prostych implementacji.

\begin{itemize}
    \item Potęgowanie haszy:
    \[
        \begin{aligned}
        f(m, i) &=
            \begin{cases}
                H(m) & \mbox{dla } i = 0 \\
                H(f(m, i - 1)) & \mbox{dla } i > 0 \\
            \end{cases}
        \\
        h &= f(m, n)
        \end{aligned}
    \]

    \item Potęgowanie haszy z~użyciem konkatenacji:
    \[
        \begin{aligned}
        f(m, i) &=
            \begin{cases}
                H(m) & \mbox{dla } i = 0 \\
                H(f(m, i - 1) || m) & \mbox{dla } i > 0 \\
            \end{cases}
        \\
        h &= f(m, n)
        \end{aligned}
    \]

    \item Potęgowanie haszy z~użyciem konkatenacji i~soli:
    \[
        \begin{aligned}
        f(m, i) &=
            \begin{cases}
                H(m) & \mbox{dla } i = 0 \\
                H(f(m, i - 1) || m || s) & \mbox{dla } i > 0 \\
            \end{cases}
        \\
        h &= f(m, n)
        \end{aligned}
    \]

\end{itemize}

Oznaczenia:
\begin{itemize}
    \item $m$ -- oryginalna wiadomość,
    \item $s$ -- sól,
    \item $h$ -- wyjściowy skrót,
    \item $H$ -- funkcja haszująca,
    \item $n$ -- liczba iteracji wydłużająca czas obliczeń.
\end{itemize}

Jedną z~nowszych implementacji tej techniki stanowi \texttt{scrypt}. Stara się
on udaremnić ataki korzystające z~wyspecjalizowanego sprzętu poprzez nałożenie
dużych wymagań obliczeniowych: atakujący albo musi zainwestować dużą liczbę
cykli procesora w~celu obliczenia ostatecznego klucza, albo musi posiadać dużo
pamięci tak, aby mógł w~niej przechować wyniki powtarzających się obliczeń.
Należy zwrócić uwagę, że ,,specjalistyczny sprzęt'' to w~większości macierze
procesorów. Wyposażenie tych procesorów w~pamięć potrzebną do przeprowadzenia
skutecznego ataku byłoby bardzo kosztowne, co w~efekcie zapewnia ochronę przed
zrównoleglonymi atakami.  Pochodne tego algorytmu są wykorzystywane
w~kryptowalutach takich jak Litecoin oraz Dogecoin.
