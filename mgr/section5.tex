\section{Ataki teoretyczne}
Poza atakami uniwersalnymi istnieje rodzina ataków wyspecjalizowanych, które
biorą na cel słabości konkretnych implementacji funkcji haszujących.

\subsection{Atak urodzinowy}
Paradoks urodzinowy polega na tym, że mając ciąg $P$ złożony z~elementów
wybranych ze stałym rozkładem prawdopodobieństwa ze zbioru $M$,
prawdopodobieństwo, że $\forall m, m' \in P \; m \neq m'$ jest dane
wzorem~\cite{birthday}:

$$ \Pr(X) = \prod_{i=1}^{|P|} \frac{|M|-i}{|M|} $$

Znajduje to zastosowanie przy atakach siłowych na funkcje haszujące.
Z~powyższej własności wynika, że dla $n$-bitowej funkcji skrótu, aby natrafić
na kolizję z~prawdopodobieństwem większym niż $p$, należy wygenerować
liczbę skrótów daną wzorem:

$$ x : \prod_{i=1}^{x} \frac{2^n-i}{2^n} \geq p $$

%Można na tej podstawie sformułować $p(n,M)$ równe prawdopodobieństwu
%wylosowania dwukrotnie tego samego elementu podczas losowania $n$ losowych
%elementów z~$H$. Funkcja ta może być wyrażona następującym przybliżeniem:

%$$p(n,H) \approx 1-e^{-n^2/(2|M|)}$$

Przybliżona wartość tej funkcji wynosi~\cite{birthday2}:

$$ x = \sqrt{2 |M| \ln{\frac{1}{1-p}}} $$

Korzystając z~tego wzoru można skonstruować tabelkę poniżej. Tabelka
przedstawia liczby haszy, które należy wygenerować, żeby otrzymać kolizje
z~prawdopodobieństwem nie mniejszym niż $k$.

\begin{tabular}{|r|c|*{5}{r|}}
\hline
\multirow{2}{*}{$n$} & \multirow{2}{*}{$|M|$} & \multicolumn{5}{c|}{$k$ (przybliżone)} \\
\cline{3-7}
& & 1\% & 10\% & 25\% & 50\% & 75\% \\
\hline
   8 &    $2^{8}$ & 3                     & 8                     & 13                    & 19                    & 27 \\
  16 &   $2^{16}$ & 37                    & 118                   & 195                   & 301                   & 426 \\
  32 &   $2^{32}$ & 9292                  & 30084                 & 49711                 & 77163                 & 109125 \\
  64 &   $2^{64}$ & $6.0 \times 10^{8}$   & $1.9 \times 10^{9}$   & $3.3 \times 10^{9}$   & $5.6 \times 10^{9}$   & $7.2 \times 10^{9}$ \\
 128 &  $2^{128}$ & $2.6 \times 10^{18}$  & $8.5 \times 10^{18}$  & $1.4 \times 10^{19}$  & $2.2 \times 10^{19}$  & $3.1 \times 10^{19}$ \\
 256 &  $2^{256}$ & $4.8 \times 10^{37}$  & $1.6 \times 10^{38}$  & $2.6 \times 10^{38}$  & $4.0 \times 10^{38}$  & $5.7 \times 10^{38}$ \\
 512 &  $2^{512}$ & $1.6 \times 10^{76}$  & $5.3 \times 10^{76}$  & $8.8 \times 10^{76}$  & $1.4 \times 10^{77}$  & $1.9 \times 10^{77}$ \\
1024 & $2^{1024}$ & $1.9 \times 10^{153}$ & $6.2 \times 10^{153}$ & $1.0 \times 10^{154}$ & $1.6 \times 10^{154}$ & $2.2 \times 10^{154}$ \\
\hline
\end{tabular}

\pagebreak
Ogólnie przyjmuje się, że aby złamać $n$-bitową funkcję skrótu, wystarczy
przeprowadzić $2^\frac{n}{2}$ operacji haszowania~\cite{birthday3}.



\subsection{\en{Collision attack}}
Gdy funkcja nie ma własności \en{collision resistance}, można łatwo znaleźć $m'
\neq m : H(m) = H(m')$, gdzie ,,łatwo'' oznacza proces o~złożoności
obliczeniowej mniejszej niż teoretyczna granica $O(2^\frac{n}{2})$
wynikająca z~opisanego wyżej paradoksu urodzinowego. Atak wykorzystujący tę
cechę znany jest jako \en{collision attack} i~stanowi najpopularniejszy
rodzaj ataków na współczesne funkcje haszujące.

Przykładowo, dla \texttt{SHA-1} teoretyczna granica wynosi $O(2^{80})$, lecz
znany jest atak wymagający jedynie $O(2^{51})$ operacji
haszowania~\cite{best_sha1_collision_attack}. Z~kolei dla \texttt{MD5} zamiast
$O(2^{64})$ potrzeba jedynie $O(2^{20.96})$
operacji~\cite{best_md5_collision_attack}.

Najskuteczniejsze ataki typu \en{collision} na kryptograficzne funkcje
haszujące są atakami różnicowymi, w~których konstruowane są dwie wiadomości $M$
i $M'$ różniące się w~pewien nieznaczny sposób. Najpopularniejsza konstrukcja
ataku, przedstawiona poniżej, stosowana jest m.in. przeciwko \texttt{MD4},
\texttt{MD5}, \texttt{RIPEMD}, \texttt{SHA-0} i~\texttt{SHA-1}. Zapoczątkowana
została ona przez Wang Xiaoyun
w~\cite{wang_collision_attack1,wang_collision_attack2,wang_collision_attack3}
i~później rozwinięta przez innych kryptologów.

\begin{enumerate}

\item Wybierany jest pewien wektor różnic między dwoma wiadomościami $M$
i~$M'$, oznaczony $\Delta M = M' - M$. Początkowo wektor ten był dobierany
ręcznie, później jednak zaczęto znajdywać sposoby jego automatycznego
doboru (przykładem może być \cite{sha1_disturbance_vector}).

\item Po wybraniu $\Delta M$ analizowany jest sposób, w~jaki bity $\Delta M$
wpływają na wynik działania funkcji haszującej. Proces ten polega na
prześledzeniu propagacji początkowych różnic pomiędzy $M$ a~$M'$ na końcowe
różnice pomiędzy $H$ a~$H'$.

\item Analiza ta pozwala wyciągnąć pewne wnioski i~sformułować szereg warunków
wystarczających, aby $M$ i~$M'$ dawały kolizję. Warunki te ustalają, jakie
cechy powinien spełniać stan wewnętrzny funkcji haszującej w~kolejnych krokach
działania. Przekłada się to w~praktyce na to, że jeśli mając losową
wiadomość $M$ stan wewnętrzny funkcji haszującej nie będzie zgadzał się ze
sformułowanymi warunkami, należy szukać innej wiadomości -- takiej, dla której
będą zachodziły wszystkie warunki.

Warunki nie są zdefiniowane jednoznacznie -- możemy formułować różne,
niezależne zbiory warunków wystarczających do zajścia kolizji. Im mniejszy
zbiór warunków, tym lepiej: większa swoboda co do stanu wewnętrznego funkcji
haszującej przekłada się na większe prawdopodobieństwo szybkiego znalezienia
kolizji, a~zatem mniejszy koszt obliczeń.

Warunki te na początku prawdopodobnie były znajdywane ręczne. Istnieją jednak
propozycje automatyzacji ich szukania; przykładem może być algorytm
zaproponowany przez Martina Schl{\"a}ffera~\cite{md4_differential_paths}.

\item Mając sformułowane warunki wystarczające do zajścia kolizji, zazwyczaj
okazuje się, że prawdopodobieństwo znalezienia $M$ dla którego są one spełnione
jest niższe niż prawdopodobieństwo znalezienia kolizji wynikające z~paradoksu
urodzinowego. W~związku z~tym wprowadzane są techniki modyfikacji wiadomości
$M$, które wykonywane są w~zależności od cech jakie spełnia stan wewnętrzny na
różnych etapach działania funkcji haszującej. Techniki te mają na celu
zwiększenie prawdopodobieństwa, że dla danego $M$ stan wewnętrzny będzie się
zgadzał ze sformułowanymi warunkami.

\item Mając techniki modyfikacji wiadomości oraz warunki wystarczające do
zajścia kolizji, można zacząć poszukiwania wiadomości $M$, która po
zastosowaniu tych technik spełni wspomniane warunki.

\item Znaleziona wiadomość $M$ oraz $M' = M +\Delta M$ daje kolizję.

\end{enumerate}

Ataki tego typu są wykorzystywane głównie do fałszowania podpisów cyfrowych.
Przykładem może być złamanie podpisu centrum certyfikacji przy pomocy macierzy
200 konsoli \enn{Playstation 3} w~2008~r.\cite{ps3_attack}, co umożliwiło
atakującym wystawienie ważnego certyfikatu dowolnej stronie WWW. Innym
przykładem może być sfałszowanie przez wirus Flame podpisu firmy
\enn{Microsoft} służącego do identyfikacji zaufanych sterowników systemu
operacyjnego\cite{flame_attack}, co umożliwiło mu wykonywanie dowolnego kodu
w~trybie jądra.

Odmianą \en{collision attack} jest \en{chosen prefix collision attack},
w~którym atakujący mając dane prefiksy $p_1, p_2$ szuka dwóch wiadomości $m_1,
m_2$ dla których $H(p_1 || m_1) = H(p_2 || m_2)$.


\subsection{\en{Preimage attack}}
Atak typu \en{preimage} dzieli się na dwie kategorie, zależnie od rodzaju
własności \en{preimage resistance}, która jest atakowana.
    \begin{itemize}

    \item W~przypadku \en{preimage resistance}, mając dany hasz $h$, atakujący
    próbuje znaleźć dowolną wiadomość $m'$, dla której $H(m') = h$.

    \item W~przypadku \en{second preimage resistance}, mając daną wiadomość $m$,
    atakujący próbuje znaleźć wiadomość $m' \neq m$,dla której $H(m) = H(m')$.

    \end{itemize}
Opisane w~poprzedniej sekcji ataki uniwersalne są przykładami
brutalnych ataków \en{preimage}.

Warto wspomnieć o~tym, że jeśli funkcja nie jest odporna na \en{collision
attack}, nie czyni to jej bezużyteczną dla kryptograficznych potrzeb. Gdy
funkcja jest odporna na \en{preimage attack}, może być z~powodzeniem
wykorzystywana we wszystkich zastosowaniach, których specyfika nie wymaga
odporności na \en{collision attack}. Przykładem takiego zastosowania może być
przechowywanie haszy w~bazie danych: nawet gdy atakujący potrafi generować $m
\neq m' : H(m) = H(m')$, nie przyda mu się to do złamania haseł istniejących
użytkowników.

Tego typu atak jest znacznie trudniejszy do przeprowadzenia w~stosunku do
\en{collision attack} z~uwagi na to, że w~obu odmianach ataku mamy narzuconą
z~góry wartość w~stosunku do której szukamy kolizji, podczas gdy \en{collision
attack} pozostawia maksymalną swobodę co do postaci $m$ i~$m'$. Swoboda ta
w~przypadku \en{collision attack} manifestuje się często np. tym, że długość
wiadomości $m$ i~$m'$ jest podzielna przez długość pojedynczego bloku funkcji
skrótu; atak \en{preimage} musi natomiast wspierać wiadomości dowolnych
długości.

Teoretyczna złożoność złamania własności \en{preimage resistance} dla
$n$-bitowej bezpiecznej funkcji haszującej wynosi $O(2^n)$. O~pomyślnym ataku
mówi się wówczas, gdy polepsza on tę złożoność sprowadzając ją do niższej
wartości. Dla \texttt{SHA-1} na chwilę obecną nie są znane żadne teoretyczne
ataki \en{preimage}, natomiast dla \texttt{MD5} opublikowany został atak
obniżający $O(2^{128})$ do $O(2^{123.4})$~\cite{best_md5_preimage_attack}.



\label{sec:length_extension_attack}%
\subsection{\en{Length-extension attack}}
Innym rodzajem ataku jest tzw. \en{length extension attack}. Dotyczy on jedynie
kryptosystemów, które udostępniają publicznie oryginalną wiadomość $m$ oraz
$H(s || m)$, gdzie $||$ jest operatorem konkatenacji, a~$s$ sekretem znanym
jedynie oryginalnemu serwerowi. Takie systemy kryptograficzne mają na celu
weryfikowanie autentyczności $m$ (czy nie pochodzi z~obcego źródła),
sprawdzając, czy $H(s||m)=h$. Bezpieczeństwo opiera się głównie na sile sekretu
$s$~-- znając $s$, można fałszować dowolne wiadomości.

Okazuje się, że funkcje korzystające z~konstrukcji takich jak konstrukcja
Merkle-Damg\r{a}rda, w~szczególności funkcje \texttt{MD5} oraz \texttt{SHA-1},
umożliwiają atakującemu dopisanie do danej wiadomości $m$ własnego sufiksu $m'$
i~wygenerowanie $h' : h' = H(s || m || m')$ (a zatem $h'$ potwierdzającego
nieprawdziwą autentyczność $m||m'$) nawet w~sytuacji gdy atakujący nie zna $s$.
Jest to możliwe dzięki temu, że funkcje tego rodzaju zwracają hasz w~postaci
ostatnio obliczonego wewnętrznego stanu, dlatego do skonstruowania $h'$
wystarczy kontynuować obliczenia tam, gdzie się wcześniej zakończyły, przy
pomocy dodatkowych bitów z~$m'$. Ataki tego rodzaju są szczegółowo opisane
w~\cite{md5_length_extension_attack}; przykładowa implementacja dla
\texttt{MD5} może być z~kolei znaleziona
w~\cite{md5_length_extension_attack_implementation}.
