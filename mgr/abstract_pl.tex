Celem tej pracy jest przedstawienie współczesnych metod łamania
kryptograficznych funkcji skrótu, zarówno od strony praktycznej, jak
i~teoretycznej. Autor skupia się na dwóch funkcjach: \texttt{MD5} oraz
\texttt{SHA-1}.

Praca złożona jest z~czterech części. Pierwsza stanowi wprowadzenie do tematyki
kryptograficznych funkcji skrótu, opisując krótko ich rys historyczny,
zgłębiając szczegółowo budowę i~przedstawiając ich zastosowania.

Druga część opisuje szereg uniwersalnych technik, które można zastosować do
atakowania większości współczesnych kryptograficznych funkcji skrótu w~sposób
niezależny od ich struktury.

Część trzecia prezentuje od strony technicznej wybrane wysoce wyspecjalizowane
ataki teoretyczne, jakie powstały na przestrzeni ostatnich lat, pokazując luki
w~kryptograficznych funkcjach skrótu.

W~ostatniej części przedstawione są metody obrony przed atakami zarówno
uniwersalnymi, jak i~wyspecjalizowanymi.
