\section{Podsumowanie}
Kryptograficzna funkcje haszujące, podobnie jak ataki na nie, cały czas
ewoluują. Zarówno użytkownicy, jak i~atakujący dysponują coraz większymi
zasobami. Wraz z~upływem czasu wzrastają możliwości obliczeniowe maszyn; ataki
brutalne stają się coraz skuteczniejsze. Jednak nawet w~takiej sytuacji
zastosowanie paru prostych technik może zagwarantować bezpieczeństwo,
wydłużając czas potrzebny na skuteczne przeprowadzenie ataku do nierealnych
wartości. Funkcje typu \texttt{MD5} i~\texttt{SHA-1} choć nieodporne na
\en{collision attack}, nadal dzielnie stawiają czoła atakom \en{preimage};
a~gdy to nie wystarcza, wystarczy sięgnąć po konstrukcje takie jak
\texttt{SHA-256} czy \texttt{SHA-3}, które jak dotychczas opierają się
wszelkim kryptoanalizom.

W~chwili obecnej atakujący są zatem zawsze krok za użytkownikami i~stan ten
będzie się utrzymywał prawdopodobnie tak długo, aż nie zostanie rozwiązany
problem $\textrm{P} \stackrel{?}{=} \textrm{NP}$.
