\section{Podsumowanie}
Funkcje haszujące, podobnie jak ataki na nie, cały czas ewoluują. Zarówno
użytkownicy, jak i~atakujący dysponują coraz większymi zasobami. Wraz z~upływem
czasu wzrastają możliwości obliczeniowe maszyn; ataki brutalne stają się coraz
skuteczniejsze. Jednak nawet w~takiej sytuacji zastosowanie paru prostych
technik może zagwarantować bezpieczeństwo, wydłużając czas potrzebny na
skuteczne przeprowadzenie ataku do nierealnych wartości. Funkcje typu
\texttt{MD5} i~\texttt{SHA-1} choć nieodporne na \en{collision attack}, nadal
dzielnie stawiają czoła atakom \en{preimage}; a~gdy to nie wystarcza, wystarczy
sięgnąć po konstrukcje takie jak \texttt{SHA-256} czy \texttt{SHA-3}, które na
chwilę obecną opierają się wszelkim kryptoanalizom.

Na chwilę obecną atakujący są zatem zawsze krok w~tył w~stosunku do
użytkowników i~wzorzec ten będzie się utrzymywał prawdopodobnie tak długo, aż
nie zostanie rozwiązany problem $\textrm{P} \stackrel{?}{=} \textrm{NP}$.
